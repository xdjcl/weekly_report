\documentclass{ctexart}
\usepackage{graphicx}
\usepackage{subfigure}
\usepackage{hyperref}
\usepackage{subfigure}
\usepackage{geometry}
\geometry{left=2.5cm,right=3.5cm,top=2.5cm,bottom=2.5cm}
\CTEXsetup[format+={\flushleft}]{section}

\begin{document}
\CJKfamily{li}
\title{周报}
\author{刘精昌}
\maketitle

\fangsong
\section*{本周工作}
\begin{enumerate}
  \item 看完MLAPP第十二章,主要介绍FA,PCA,ICA。
  \item 继续了解CNN相关知识。
    \begin{itemize}
      \item 看完UFLDL在线教程。
      \item 阅读《A Taxonomy of Deep Convolutional Neural Nets for Computer Vision》,主要围绕CV,介绍了一些CNN结构以及改进。首先介绍了CNN的组件、训练算法、精确度提升等基本知识,其次介绍了用于图像识别、语义对象分割与解析、多模式网络,带有RNN结构,动作识别等各种情况下的CNN结构。该文可以作为一个进一步了解CNN的参考。
    \end{itemize}
  \item 阅读《Automatic Construction of Nonparametric Relational Regression Models for
Multiple Time Series》ICML16。该文提出了学习高斯过程的Compositional Kernel的改进算法。Compositional Kernel提出于《Structure Discovery in Nonparametric Regression through Compositional Kernel Search》ICML13,Compositional Kernel主要是对Base Kernel 做加、乘、change-point、change-window操作得到。该改进算法主要基于Automatic Bayesian Covariance Discovery (ABCD),ABCD由《Automatic Construction and Natural-Language Description of Nonparametric Regression Models》ICML14提出。该改进算法主要是遍历由Base Kernel扩张而来的Compositional Kernels集合,先利用极大似然估计学习参数,再利用BIC准则选择得到Compositional Kernel。改进方面需要进一步看相关论文。

\end{enumerate}

\section*{下周计划}
\begin{itemize}
  \item 主要学习高斯过程、Kernel相关知识:看MLAPP相关章节以及Compositional Kernel相关论文。
\end{itemize}

\end{document} 