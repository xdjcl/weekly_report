\documentclass{article}
\usepackage{ctex}
\usepackage{hyperref}
\usepackage{fontspec}
\usepackage{hyperref}

\setmainfont{FangSong}

\begin{document}

\title{周报}
\author{刘精昌}
\maketitle

\section*{本周工作}
\begin{enumerate}
  \item 借助\href{http://www.seas.ucla.edu/~vandenbe/ee236c.html}{UCLA的优化教程}详细地看了优化的一些假定(lipschitz连续、强凸等),理解了Lipschitz continuity 导数、共轭函数的凸性、co-coercivity、二次上界这些定义的等价性。
  \item 结合Bottou等人在ICML16的\href{http://users.iems.northwestern.edu/~nocedal/ICML}{tutorial},看了他前不久的出版物\href{https://arxiv.org/abs/1606.04838}{《Optimization Methods for Large-Scale Machine Learning》}前面关于SGD的部分,主要包括优化问题起源以及SGD部分。SGD部分详细地阐述了固定步长下,SGD能做到线性收敛到最优点附近点;在衰减步长的情形下,SGD次线性收敛到最优点。文章具有一定的深度,作者站在比较高的角度对问题进行阐述,分析,值得好好阅读、理解。
  \item 粗略看了一些关于SGD改进(SAG、SAGA)的文章。后面会继续看、思考、整理。
\end{enumerate}

\section*{后续计划}
\begin{itemize}
    \item 继续看Bottou的文献后面关于SGD改进的部分,继续看、理解一些讲SGD改进的文章,然后对这些进行总结。
\end{itemize}
\end{document} 