\documentclass{ctexart}
\usepackage{graphicx}
\usepackage{subfigure}
\usepackage{hyperref}
\usepackage{subfigure}
\usepackage{geometry}

\begin{document}

\title{周报}
\author{刘精昌}
\maketitle

\fangsong
\section*{本周工作}
\begin{enumerate}
  \item 结合提出SAGA、SVRG的经典论文,继续阅读Bottou等人的\href{https://arxiv.org/abs/1606.04838}{《Optimization Methods for Large-Scale Machine Learning》}Noise Reduction Methods、Other Popular Methods两部分。SGD方法每次迭代用单个梯度替代全梯度,因为单个梯度并不是全梯度,也就是单个梯度替代全梯度有一定的noise。这种noise的存在,导致SGD的收敛慢,于是可以考虑reduce noise。Paper中总结了三种noise reduce的方法:Dynamic Sample Size Methods、Gradient Aggregation、Iterate Averaging Methods。其中SAGA、SVRG这些经典的SGD变种就属于从Gradient Aggregation角度reduce noise。
  \item 和亦锬师兄联系。浏览了他提供的两篇paper。一篇主要是解决model异步并行的$f(x)+g(x)$优化问题。下面会继续看。
\end{enumerate}

\section*{后续计划}
\begin{itemize}
    \item 亦锬师兄推荐的那两篇文章相关。
\end{itemize}
\end{document} 