\documentclass{ctexart}
\usepackage{graphicx}
\usepackage{subfigure}
\usepackage{hyperref}
\usepackage{subfigure}
\usepackage{geometry}
\geometry{left=2.5cm,right=3.5cm,top=2.5cm,bottom=2.5cm}
\CTEXsetup[format+={\flushleft}]{section}

\begin{document}
\CJKfamily{li}
\title{周报}
\author{刘精昌}
\maketitle
\fangsong

\section*{本周工作}
\begin{enumerate}
  \item 准备组会报告
  \item 阅读两篇优化相关的paper,《Asynchronous Multi-Task Learning》和亦锬师兄的《Make Workers Work Harder: Decoupled Asynchronous Proximal Stochastic Gradient Descent》。两者主要是在在求解$\mathop {\min }\limits_x f\left( x \right) + g\left( x \right)$, $f\left( x \right) \buildrel \Delta \over = \frac{1}{n}\sum\limits_{i = 1}^n {\mathop f\nolimits_i \left( x \right)} $,$g(x)$通常凸但是不光滑。求解这样的优化问题,需要prox操作,prox操作的复杂度较大。《Asynchronous Multi-Task Learning》中将prox操作放在server上进行,client负责计算梯度。亦锬师兄将prox分散到client上进行,能减少server上的计算延迟。。
\end{enumerate}

\section*{下周计划}
\begin{itemize}
  \item 主要以看优化书籍为主。
\end{itemize}
\end{document} 