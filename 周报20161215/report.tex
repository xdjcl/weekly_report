\documentclass{article}
\usepackage{ctex}
\usepackage{hyperref}

\begin{document}

\title{周报}
\author{刘精昌}
\maketitle

\section*{本周工作}
\begin{enumerate}
  \item 看完《Distributed Optimization and Statistical Learning via the Alternating Direction Method of Multipliers》(ADMM),看了一点《convex optimization》内容。
  \item 看了Bach等在NIPS16报告的tutorial《Stochastic optimization: Beyond stochastic gradients and convexity》,主要是对最近优化工作的总结,有较大启发。目前的优化关注于4个问题:
      \begin{enumerate}
        \item 有限次项求和问题,主要工作是各种SGD的推广:SAG、SDCA、SARG、SVRG、ADAM等
        \item 有限次求和加正则化项,主要是prox相关算法。
        \item 非凸优化问题,和凸优化问题比较本质的区别是证明收敛性等时,凸优化中的一些性质不能直接使用,且证明目标和凸优化有一定区别。
        \item 分布式优化问题,近年来有很多工作围绕SARG、SVRG等的分布式展开。
      \end{enumerate}
  \item 看了关于两篇SGD变形的papers:《A Stochastic Gradient Method with an Exponential Convergence Rate for Finite Training Sets》(SAG)nips12和《Accelerating Stochastic Gradient Descent using Predictive Variance Reduction》(SVRG)nips13,SAG和SVRG提升了SGD的收敛率。比较经典的两篇文章,把SGD的次线性收敛率提升到线性,SVRG相对于SAG的改进在于不需要再记录下之前的梯度,论文里说它更适合神经网络训练。
  \item 准备本周组会报告。
\end{enumerate}

想了下《Distribution multi-task learning》AISTST16,认为该工作解决MTL问题有一定的针对性,不general,而且不能做到隐私保护。

\section*{下周计划}
\begin{itemize}
    \item 看其他关于SGD改进文章。
    \item 考虑分布式MTL问题,考虑请教师兄,听听师兄看法。
    \item 花时间应对考试
    \item 发周报前突然看见《Distribution multi-task learning》作者挂在Arxiv上的文章 \href{https://arxiv.org/abs/1603.02185}{Distributed Multi-Task Learning with Shared Representation》} ,里面提到了很多方法,包括ADMM,周五仔细看看。
\end{itemize}
\end{document} 